% This is a sample document using the University of Minnesota, Morris, Computer Science
% Senior Seminar modification of the ACM sig-alternate style to generate a simple annotated
% bibliography. The idea is that this document is fairly short, consisting of a brief description
% of your sources and how you intend to use them (or not). Most of the ``content'' of the
% generated document comes from the bibliography file, including the notes field which will
% provide the annotations.

% See https://github.com/UMM-CSci/Senior_seminar_templates for more info and to make
% suggestions and corrections.

\documentclass{sig-alternate}

\usepackage{url}

\begin{document}

% --- Author Metadata here ---
%%% REMEMBER TO CHANGE THE SEMESTER AND YEAR AS NECESSARY
\conferenceinfo{UMM CSci Senior Seminar Conference, October 2020}{Morris, MN}

\title{Recent advances in smartphone computational photography}

\numberofauthors{1}

\author{
% The command \alignauthor (no curly braces needed) should
% precede each author name, affiliation/snail-mail address and
% e-mail address. Additionally, tag each line of
% affiliation/address with \affaddr, and tag the
% e-mail address with \email.
\alignauthor
Paul Friederichsen\\
	\affaddr{Division of Science and Mathematics}\\
	\affaddr{University of Minnesota, Morris}\\
	\affaddr{Morris, Minnesota, USA 56267}\\
	\email{fried701@morris.umn.edu}
}

\maketitle

\begin{abstract}
My topic is recent advances in the computational photography used in smartphones. Specifically looking at the use of multi-frame algorithms to increase image resolution and quality. I'm mainly looking at a couple of papers by researchers at Google on algorithms used in the Super-Res Zoom and Night Sight features of the Google Pixel series of smartphones.
\end{abstract}

\section{Discussion of sources}


\subsection{Primary Sources}

My primary two sources are \cite{Wronski2019, Liba2019}.
\begin{itemize}
\item Wronski 2019 \cite{Wronski2019} describes an algorithm to use natural hand tremor to create higher resolution images from multi-frame bursts. It has the effect of improving image resolution and noise performance.

\item Liba 2019 \cite{Liba2019} describes a mobile camera solution leveraging existing burst imaging techniques and improving low-light performance. It builds upon work done in \cite{Hasinoff2016} and uses the burst merging algorithm from \cite{Wronski2019}.
\end{itemize}
Both of these primary sources are long and very detailed. They are written by researchers at Google.

\subsection{Background Sources}
\begin{itemize}
\item Hasinof 2016 \cite{Hasinoff2016} is cited in both of the previous mentioned articles and describes an earlier multi-frame burst merging algorithm. It is used as a point of comparison by both primary sources and will be necessary background information. It is also from Google Research. I think this could end up being more of a primary source.
\item Barron 2017 \cite{Barron2017} is on the auto-white-balance algorithm used in \cite{Liba2019} and will be used for background information on that.
\item Wikipedia articles \cite{wiki:BayerFilter, wiki:Demosaicing, wiki:Aliasing} provide background information on Bayer filters, demosaicing, and aliasing that are necessary to explain how the algorithm in \cite{Wronski2019} improves on existing techniques.

\item \cite{blog:Wronski2018, blog:Levoy2018} are the blog posts accompanying \cite{Wronski2019, Liba2019} respectively. They provide additional information and examples of the algorithms.

\end{itemize}
% The following two commands are all you need to
% produce the bibliography for the citations in your paper.
\bibliographystyle{abbrv}
% annotated_bibliography.bib is the name of the BibTex file containing 
% all the bibliography entries for this example. Note that you *don't* include the .bib ending
% in the \bibliography command.
\bibliography{paper}  

% You must have a ".bib" file and remember to run:
%     pdflatex bibtex pdflatex pdflatex
% in order to see all the citation references correctly.

\end{document}

